\section{Checking violation hadling}

\begin{frame}[t]{Questions}
\begin{itemize}
  \item \textbf{Q}:
        How do we select a fail-fast with no further execution?
    \begin{itemize}
      \item \textbf{Answer}:
            This is equivalent to an empty handler marked as \cppkey{[[noreturn]]} or perhaps calling to \cppid{terminate()}.
    \end{itemize}

  \vfill\pause
  \item \textbf{Q}:
        How do we select other handler?
    \begin{itemize}
      \item \textbf{Answer}:
      This is equivalent to a non-empty handler marked as \cppkey{[[noreturn]]} or we can define that after the handler \cppid{terminate()} is invoked.
    \end{itemize}

  \vfill\pause
  \item \textbf{Q}:
        How do we select throwing an exception?
    \begin{itemize}
      \item \textbf{Answer}:
            This would only lead to throwing a single global exception. This is what Ada2012 does. If we take the route D, this is not needed.
    \end{itemize}
\end{itemize}
\end{frame}

\begin{frame}[t]{Questions}
\begin{itemize}
  \item \textbf{Q}:
        Do fail-fast or handler require different user code?
    \begin{itemize}
      \item \textbf{Answer}:
            No. In both cases they would be noexcept operations. Consequently, \cppkey{noexcept}\cppid{(vec[i])} would be true all the time.
    \end{itemize}

  \vfill\pause
  \item \textbf{Q}:
        What if I want to throw?
    \begin{itemize}
      \item \textbf{Answer}:
            You may option Option D and specify the exception you want to throw which is a design decision.
    \end{itemize}
\end{itemize}
\end{frame}
