\section{Summary}

As a summary, I explore different options to programmer in source code and the
effects with each build mode.

\subsection{Contract options}

A programmer may express a contract as one of the following:

\begin{itemize}
  \item \textbf{Non-throwing contract}: A contract that guarantees not to
throw. A violation of such contract always results in some sort of program
termination.

  \item \textbf{Potentially throwing contract}: A contract that might throw when
the application is built in the appropriate mode.

  \item \textbf{Throwing contract}: A contract where the programmer specifically
requests to throw. The fact that an exception is thrown or not is still
dependent on the specific build mode used.
\end{itemize}

\subsubsection{Non-throwing contract}

\begin{lstlisting}
template <typename T>
T & myvector::operator[](int pos) noexcept
[[expects: 0<=pos && pos < size()]]
{
  return buf[pos];
}
\end{lstlisting}

\subsubsection{Potentially throwing contract}

\begin{lstlisting}
template <typename T>
T & myvector::operator[](int pos) 
[[expects: 0<=pos && pos < size()]]
{
  return buf[pos];
}
\end{lstlisting}

\subsubsection{Throwing contract}

\begin{lstlisting}
template <typename T>
T & myvector::operator[](int pos) 
[[expects: 0<=pos && pos < size(), effect: throw bounds_error{pos,size()}]]
{
  return buf[pos];
}
\end{lstlisting}

\subsection {Build modes}

The following modes are considered:

\begin{itemize}
  \item \textbf{failfast}. The application terminates without further cleanup
(e.g. calling \cppid{quick\_exit()}).
  \item \textbf{terminate}. The application terminates invoking the
corresponding \cppid{terminate\_handler}.
  \item \textbf{resume}. The application invokes a system wide handler function
which then may return to resume executing despite the contract violation.
  \item \textbf{throw}. The application throws a system wide exception. This
exception could be supplied to the build system or just a global predefined one
(i.e. \cppid{broken\_contract}).
  \item \textbf{throw-specific}. The application throws the exception specified
in source code.
\end{itemize}

\subsubsection{Terminating modes}

Modes \emph{failfast} and \emph{terminate} are essentially the same and,
consequently, are studied here together. The only difference is that for
\emph{terminate} mode, the termination includes the execution of a user supplied
handler and proper cleanup.

\begin{itemize}
  \item Non-throwing contracts: The program terminates.
  \item Potentially throwing contracts: The program terminates without invoking
any exception.
  \item Throwing contracts: The program terminates without invoking the user
supplied exception.
\end{itemize}

\subsubsection{Resume mode}

Mode \emph{resume} results in the execution of a user provided handler. The
handler is supplied to the build system, and it is global for the whole
application.

\begin{itemize}
  \item Non-throwing contracts: The program runs the resume handler but does not
resume. Instead it invokes the \cppid{terminate} function.
  \item Potentially throwing contracts: The program runs the resume handler and
after that, it proceeds execution.
  \item Throwing contracts: The program runs the resume handler and after that,
it throw the exception supplied by the contract.
\end{itemize}

\subsubsection{Throwing mode}

Mode \emph{throw} results in throwing a global exception. This might be a
predefined exception (e.g. \cppid{broken\_contract}) or an exception supplied to
the build system. In any case, the exception would be the same one for the whole
application.

\begin{itemize}
  \item Non-throwing contracts: The program ignores the global exception and
terminates by invoking the \cppid{terminate} function.
  \item Potentially throwing contracts: The global exception is thrown.
  \item Throwing contracts: The program ignores the global exception and the
exception specified by the throwing contract is thrown.
\end{itemize}

\subsubsection{Throwing specific mode}

Mode \emph{throw-specific} results in throwing the exception specified by the
contract if any.

\begin{itemize}
  \item Non-throwing contracts: The program 
terminates by invoking the \cppid{terminate} function.
  \item Potentially throwing contracts: A global exception is thrown.
  \item Throwing contracts: The 
exception specified by the throwing contract is thrown.
\end{itemize}

Consequently, the modes \emph{throw} and \emph{throw-specific} can be merged
into a single mode.
