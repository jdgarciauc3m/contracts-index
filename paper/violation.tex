\section{Effects on contract violation}

Multiple effects have been suggested when a contract is violated. It is extremely
important to remark that the set of valid effects is highly dependent on the type
of applications being used.

In general the fact that a contract has been violated is a symptom that the
program has a defect. That is, a contract is not violated because of any
exogenous circumstance, but because the program is incorrectly using a function.

\subsection{Fail-fast and termination}

For some systems, once a defect has been detected the only thing that can be
done is terminating the program. This is specially true for some safety critical
applications and some real time systems.

A minor difference here is whether the system can eventually perform some final
operations in the form of a \cppid{terminate()} handler.

\subsection{Handle and continue use case}

There are use cases where what the application developer wants is to handle is
some way the contract violation and the proceed with the program. A use case
that has bee repeatedly reported by Bloomberg is the case where upon contract
violation you may want to log the contract violation and proceed with the
program execution.

\subsection{Recovering from a software defect}

There are also systems where after a contract violation the system may recover
and continue. One typical example is when a program as a structure where
plug-ins may be added to the application. In those cases, a defect in a certain
plug-in should not stop the whole application but the specific faulty plug-in.

Another example are applications that after a contract violation, still may want
to cleanly save their state to a file and perform a regular cleanup.

\subsection{Contracts and testing frameworks}

Several unit testing frameworks use exceptions to notify that a test case has
failed. This has been used sometimes to argue in favor of exceptions use to
notify contract violation. However, a test case failure is different from a
contract violation.

In general, with some given testing approaches it is relevant to be able to
throw an exception when a contract violation happens. This would allow to test
the contracts themselves, invoking functions with unexpected inputs to check
that the contract is able to correctly detect the violation.

\subsection{Summary of options}

Given this plethora of different use cases, the following options seem to be
needed by some subset of users:

\begin{itemize}
  \item Be able to perform a fail-fast. Probably this could be achieved by just
invoking \cppid{abort()}.
  \item Be able to terminate the program invoking the corresponding
\cppid{terminate\_handler()}.
  \item Be able to invoke a handler function and the resume execution at the
calling site.
  \item Be able to report the violation by throwing an application wide
exception (e.g. \cppid{contract\_violation}).
  \item Be able to report the violation by throwing a specified exception (e.g.
\cppid{out\_of\_bounds}).
\end{itemize}

This different behaviors could be specified by a build option. However, library
writers still need the ability to specify two families of contracts:

\begin{enumerate}

  \item \textbf{Non-throwing contracts}: Contracts that independently of the
build mode used will never throw. Those contracts are essential to be able to
specify preconditions and postconditions for operations that still want to
provide the \cppkey{noexcept} guarantee. The only valid effects for the
violation of such contracts are some form of termination.

  \item \textbf{Potentially throwing contracts}: Contracts that depending on the
build mode may throw an exception. For these contracts if the build mode is to
report throw exceptions the effect is throwing an exception. However, if the
build mode is termination, they just terminate without throwing the exception.

\end{enumerate}
