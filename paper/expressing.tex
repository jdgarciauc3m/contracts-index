\section{Expressing contracts}

\subsection{Existing practice}

I start by examining \cppid{vector::}\cppkey{operator}\cppid{[]} in some popular
implementation of the standard library. The code is roughly the following:

\begin{lstlisting}
template <typename T>
T & vector::operator[](size_t pos) noexcept
{
  return buf + pos;
}
\end{lstlisting}

Firstly, the operation is marked as \cppkey{noexcept}. Even, if the standard
does not require the operation to be marked as \cppkey{noexcept},
implementations are given the freedom to add this specifier. Secondly, the
operation never throws. In fact, the source code for the debug version is
slightly different and it lives in a different source file (as the the debug
version of the data structure is also different).

\begin{lstlisting}
template <typename T>
T & vector::operator[](size_t pos) noexcept
{
  check_subscript(pos);
  return buf + pos;
}
\end{lstlisting}

\subsection{A simple contract}

With contracts a class my express its expectations through a precondition.

\begin{lstlisting}
template <typename T>
T & myvector::operator[](int pos) noexcept
[[expects: 0 <= pos && pos < size()]] 
{
  return buf[pos];
}
\end{lstlisting}

In this example the contract is part of the function declaration and is visible
to its callers. A key derived property is that the precondition is available to
the caller and consequently to any static analysis tool even if the
implementation code is not available. In this specific case, as the code is a
\cppkey{template} the source code is always available. However, there might be
cases for non-templated code where the implementation is only distributed in
binary format.

\subsubsection{Effects on contract violation}

When the build mode is terminating (fail-fast or terminate) the precondition
violation implies program termination. 

For a resuming handler build mode, the effect in this case would be rather catastrophic as
the operation would then be returning a dangling reference.

Finally, in a the case of a throwing build mode, the effect would be equally
termination as the operation has been marked as \cppkey{noexcept}.

\subsection{Implementation contracts}

A different strategy would be using contracts in function bodies. There might be
multiple reasons for a developer to do so. Firstly, some contracts might be
expressed with a lower complexity as the algorithm proceeds. Secondly, an
implementation contract may make use of implementation details not available in
the declaration.

\begin{lstlisting}
template <typename T>
T & myvector::operator[](int pos) noexcept
{
  [[check: 0 <= pos && pos < size]] 
  return buf[pos];
}
\end{lstlisting}

As it has been stated earlier this approach would limit the static analysis in
the case of non-inlined non-templated code.

\subsubsection{Effects on contract violation}

The effects would be here essentially the same that for the previous case of
contracts in the declaration. The key point is that operation is \cppkey{noexcept}.

\subsection{A throwing index operator}

Another option to be considered is that I might want to have an index operator
which throws when the index is out of bounds.

\begin{lstlisting}
template <typename T>
T & myvector::operator[](int pos)
{
  if (0 > pos || pos >= size) throw bounds_error{};
  return buf[pos];
}
\end{lstlisting}

In this case, the operation is not marked to be \cppkey{noexcept} (i.e., it is
\cppkey{noexcept(false)}). Obviously, this approach has drawbacks. The check is
now unconditional (which is asking to be controlled by some macro). Besides, the
precondition is now hidden from any static analysis tool.
