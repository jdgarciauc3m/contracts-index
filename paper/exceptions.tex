\section{Being able to throw}

As previously stated being able to throw on contract violation is essential for
some approaches, while cannot be afforded in some other use cases. 

\subsection{Throwing contract}

A possible solution is to be able to express a throwing contract. Following our
example:

\begin{lstlisting}
template <typename T>
T & myvector::operator[](int pos)
[[expects(bounds_error{}): 0<=pos && pos <=size()]]
{
  return buf[pos];
}
\end{lstlisting}

While this was the syntax originally suggested during the 2015 Kona meeting, a
more generalized one can be achieved by adding an \cppkey{effect} attribute.

\begin{lstlisting}
template <typename T>
T & myvector::operator[](int pos)
[[expects: 0<=pos && pos <=size(), effect:throw bounds_error{pos,size}]]
{
  return buf[pos];
}
\end{lstlisting}

Regardless the syntax, whenever there is a precondition violation the exception
is thrown. In this case the operation is not marked as \cppkey{noexcept}.

\subsubsection{Effects on contract violation}

When the build mode is terminating (fail-fast or terminate), the precondition
violation still implies program termination. Although this might seen as
counter-intuitive it is essential to support those application domains that
cannot afford to make use of exceptions.

When the build mode is a throwing build mode an exception will be thrown. In the
case of specified exception mode, the specific \cppid{bounds\_error} exception
is thrown. If the application is built with an application wide exception mode,
then that exception takes precedence over the specific one.

If the application is built with the resuming handler mode, then the handler is
executed and the exception is thrown, preventing the application to really
resume.
