\section{Introduction}

During the 2015 Kona meeting, Herb Sutter made three very interesting questions
on contracts in order to facilitate discussions and make progress on a contracts
proposal.

The basic idea was to explore how could a proposal be used to eventually write a
bounds check contract on the subscript operation. That is, given the a potential
\cppid{myvector} class:

\begin{lstlisting}
template <class T>
class myvector {
  //...
  T & operator[](size_t pos);
  // ...
private:
  size_t size;
  T * buf;
};

template <class T>
T & myvector::operator[](size_t pos)
{
  return buf[pos];
}
\end{lstlisting}

We want to answer the following questions:

\begin{enumerate}
  \item How do I express a range check that \cppid{0 <= pos
< }\cppkey{this}\cppid{->size} ?
  \item How do I decide whether the function should be noexcept?
  \item How do I select the whole program violation handling?
\end{enumerate}

This paper explores possible answers to this questions using the proposed
example to explore all of them. Examples are expressed following the lines
of syntax from \cite{n4415}.

In particular, I pay attention to the tension between different uses of the
contracts facility in different domains and its relationship to exceptions management.


